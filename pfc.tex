% Opciones empleadas:
%
%	a4paper -> indica el tama�o del papel, en este caso A4.
%	11pt -> tama�o de la fuente 11 puntos.
%	oneside -> s�lo escribimos en una cara del folio.
% 
% Otras opciones interesantes:
%
%	twoside -> escribimos a doble cara.
%	openbib -> para que las referencias bibliogr�ficas tengan un salto de l�nea entre cada campo de la referencia.
%
\documentclass[a4paper,12pt,twoside]{book}

\usepackage[a4paper, top=4cm, bottom=4cm, left=3.3cm, right=3.2cm]{geometry}

% Para eliminar la cabecera de las páginas vacías al final de los
% capítulos
\makeatletter
\def\cleardoublepage{\clearpage\if@twoside \ifodd\c@page\else
  \hbox{}
    \thispagestyle{empty}
      \newpage
        \if@twocolumn\hbox{}\newpage\fi\fi\fi}
\makeatother
%%

% codificaci�n latin1 y s�mbolos del idioma espa�ol (�, acentos, ...)
\usepackage[galician]{babel}
\usepackage[utf8]{inputenc}

% puede que queramos usar el s�mbolo del euro.
\usepackage{eurosym}

% El paquete fancybox nos permite crear cajas de diferentes estilos con facilidad.
% http://www.ctan.org/get/macros/latex/contrib/fancybox/fancybox.pdf
% http://www.mackichan.com/index.html?techtalk/487.htm~mainFrame
\usepackage{fancybox}
\usepackage{fancyvrb}
\usepackage{multicol}
\usepackage{amsmath}

% Para incluir subfiguras.
\usepackage{subfigure}

%Coding
\usepackage{listings}

% Para incluir gr�ficos en JPG => compilar con pdflatex.
% \usepackage[pdftex]{graphicx}

% Para incluir gr�ficos EPS => compilar con latex.
\usepackage{graphicx}

% Para escribir en color...
%
% ... cuando compilamos con el comando ``latex''
%\usepackage[dvips,usenames]{color}
% ... uando compilamos con el comando ``pdflatex''
\usepackage[pdftex,usenames,dvipsnames]{color}

% Espaciado y ajuste de m�rgenes
\usepackage{setspace}
\onehalfspacing
% \doublespacing
%\setlength{\textwidth}{14cm}
%\setlength{\textheight}{22cm}

% Paquete fancyhdr -> Para modificar la cabecera y pie de p�ginas.
% http://tug.ctan.org/tex-archive/macros/latex/contrib/fancyhdr/
\usepackage{fancyhdr}
\pagestyle{fancy}
\fancyhf{}
\fancyhf[HR]{\thepage}
\fancyhf[HL]{\nouppercase\rightmark}

% Package booktabs -> Para mejorar el aspecto de las tablas o cuadros.
% http://www.ctan.org/tex-archive/macros/latex/contrib/booktabs/
\usepackage{booktabs}

% Package rotating -> Para poder girar las tablas y dibujarlas a lo largo
% del folio en vez de a lo ancho.
\usepackage{rotating}

% Packages multicol y multirow, para manejar tablas de filas y columnas m�ltiples.
\usepackage{multicol}
\usepackage{multirow}

% Para links url
\usepackage{url}
%\usepackage{hyperref}
\usepackage[colorlinks=true,urlcolor=black,linkcolor=black,citecolor=blue,urlcolor=blue]{hyperref}

% Personalizamos la separaci�n entre p�rrafos...
\parskip=6pt

% Personalizamos el identado en la primera l�nea del nuevo p�rrafo...
\parindent=10pt

% Establecemos el n�mero m�ximo de niveles de profundidad en las secciones.
\setcounter{secnumdepth}{3}

% T�tulo
\title{...}
% Autor
\author{...}
% Fecha
\date{\today}

%Configuración para listing
\definecolor{dkgreen}{rgb}{0,0.6,0}
\definecolor{gray}{rgb}{0.5,0.5,0.5}
\definecolor{mauve}{rgb}{0.58,0,0.82}
\lstset{frame=tb,
	language=Java,
	aboveskip=3mm,
	belowskip=3mm,
	showstringspaces=false,
	columns=flexible,
	basicstyle={\small\ttfamily},
	numbers=none,
	numberstyle=\tiny\color{gray},
	keywordstyle=\color{blue},
	commentstyle=\color{dkgreen},
	stringstyle=\color{mauve},
	breaklines=true,
	breakatwhitespace=true,
	tabsize=3
}

\begin{document}

	% \maketitle sirve para generar autom�tica una portada predefinida, pero para un proyecto fin de carrera
	%	de FIC no sirvir�a porque no cumple las normas de presentaci�n. Podemos hacer dos cosas:
	% 1. Usarla e ignorar las normas (y asumir las consecuencias que pueda tener)
	% 2. Hacernos una portada en LaTeX que cumpla las normas (menos arriesgado)
	%
        %
% Portada.
%

% Nota: Sería más cómodo emplear el comando \maketitle que genera una portada de forma automática, pero 
% no incluye toda la información que es necesario incluir en la memoria de un proyecto de fin de carrera
% de la Facultad de Informática de A Coruña.
%

\begin{titlepage}

	\begin{center}

		% Logotipo de la universidad.
		\includegraphics[width=10cm]{./Images/logoudc.png}
		\vspace{2cm}

		% Nombre de la facultad, de la universidad y del departamento en que se realiza el PFC.
		{\Large{\textbf{Facultade de Informática da Universidade da Coruña}}}
		\\
		{\it \large{\textbf{Departamento de ...}}}
		\vspace{1cm}

		% Indicamos el nombre de la titulación oficial que hemos cursado con tanto esfuerzo.
		{\large TRABALLO DE FIN DE MESTRADO\\ MESTRADO UNIVERSITARIO EN ENXEÑARÍA INFORMÁTICA}
		\vspace{1cm}

		% Título
		\textbf{\Large ....}
		\vspace{7cm}
	\end{center}

	\begin{flushright}
		\begin{tabular}{ll}
			% Nombre del alumno.
			\large{\textbf{Alumno:}}	&
			\large{...} \\

			% Nombre del director del proyecto.
			\large{\textbf{Director:}}	&
			\large{...} \\

			% Fecha.
			%\large{\textbf{Fecha:}}	&
			%\large{\today} \\
		\end{tabular}
	\end{flushright}

\end{titlepage}

		
		% Insertamos una página en blanco y sin número 
		\newpage
		\mbox{}
		\thispagestyle{empty}

	% FRONTMATTER: TOC, LOF, LOT y descripci�n/organizaci�n de la memoria.
        \frontmatter

	% Los proyectos de fin de carrera de FIC han de ir acompa�ados de una serie de documentos adicionales, algunos
	% 	de ellos obligatorios (certificado, resumen, lista de palabras clave) y otros opcionales (dedicatoria
	%	y agradecimientos).
	%
		%
% Info proyecto
%
\section*{Datos do proxecto}


{\it Título do proxecto}
\begin{center}
	...
\end{center}

{\it Clase de proxecto}
\begin{center}
	...
\end{center}

{\it Nome do alumno}
\begin{center}
	...
\end{center}

{\it Nome do director}
\begin{center}
	...
\end{center}



		\thispagestyle{empty}
	
	\newpage
	\mbox{}
	\thispagestyle{empty}
	
		%
% Certificación
%
\section*{}

\begin{center}
	Don \textsc{...} profesor da Facultade de informática de A Coruña e membro do Departamento ...

\end{center}


\vspace{3cm}

	CERTIFICA: Que a memoria titulada {\it ....} foi realizada por
	\textsc{...} baixo a súa dirección e constitúe o seu Traballo Fin de
	Mestrado do Mestrado Universitario en Enxeñaría Informática.
	
	\vspace{3cm}
	
	En A Coruña, a {\today}
	
\vspace{2cm}

\begin{center}
	Don \textsc{...}
	
	Director do proxecto
\end{center}
		\thispagestyle{empty}
	
	\newpage
	\mbox{}
	\thispagestyle{empty}
	
        %
% Dedicatoria
%
\section*{}

\begin{flushright}
	{\it A .....}
\end{flushright}
        \thispagestyle{empty}     % No number page, headings...
        
    \newpage
    \mbox{}
    \thispagestyle{empty}
    
        %
% Agradecimientos
%

\section*{Agradecementos}

A \textsc{....} (membro do Departamento de ...), por ....


        \thispagestyle{empty}     % No number page, headings...
        
    \newpage
    \mbox{}
    \thispagestyle{empty}
    
        %
% Resumen del proyecto de fin de carrera
%

\section*{Resumo:}

...


\textbf{Palabras clave:} ... 
        \thispagestyle{empty}     % No number page, headings...
        
        %\include{pfc_palabras_clave}
        %\thispagestyle{empty}     % No number page, headings...
        
        %\include{pfc_certificado}
        %\thispagestyle{empty}     % No number page, headings...

        \tableofcontents
        \listoffigures
        \listoftables

        %
% Frontmatter - Introducción. Los miembros del tribunal que juzgan los PFC's tienen muchas más memorias que leer, por lo que
%	agradecerán cualquier detalle que permita facilitarles la vida. En este sentido, realizar una pequeña introducción,
%	comentar la organización y estructura de la memoria y resumir brevemente cada capítulo puede ser una buena práctica
%	que permita al lector centrarse fácilmente en la parte que más le interesa.
%

\chapter[Introdución]{
	Introdución
	\label{ch.ntr}
}

....\\

\textbf{Motivación}

....\\

\textbf{Obxectivos}

O principal obxectivo deste proxecto é ...

Estes obxectivos pódense descompoñer nos seguintes puntos:
\begin{itemize}
	\item ...
	
	\item ...
	
	
\end{itemize}

....\\

\textbf{Proposta}

Para abordar estes obxectivos proponse un sistema .....\\

\textbf{Estrutura do traballo}

Esta memoria do Traballo Fin de Mestrado estrutúrase en ... capítulos.

\begin{itemize}
	\item No capítulo .... faise un .....

\end{itemize}


	% MAINMATTER: El contenido, cap�tulo a cap�tulo, de la memoria del PFC.
        \mainmatter

	%
% TÍTULO DEL CAPÍTULO
%
\chapter[Estado da cuestión]{
	Estado da cuestión
	\label{ch.eda}
}

%
% SECCION - Título de la sección
%
\section[Introdución]{
	Introdución}


	%
% TÍTULO DEL CAPÍTULO
%
\chapter[Metodoloxía]{
	Metodoloxía
	\label{ch.mtd}
}


%
% SECCION - Título de la sección
%
\section[Metodoloxía de desenvolvemento]{Metodoloxía de desenvolvemento}

\section[Xestión de riscos]{Xestión de riscos}

\section[Xestión da configuración]{Xestión da configuración}





	%
% TÍTULO DEL CAPÍTULO
%
\chapter[Fundamentos tecnolóxicos]{
	Fundamentos tecnolóxicos
	\label{ch.fts}
}

%
% SECCION - Título de la sección
%
\section[Introdución]{
	Introdución}

\section[...]{
	...}

	%
% TÍTULO DEL CAPÍTULO
%
\chapter[Planificación, recursos e custos]{
	Planificación, recursos e custos
	\label{ch.prc}
}

%
% SECCION - Título de la sección
%
\section[Recursos]{
	Recursos}

\subsection{Recursos humanos}

\subsection{Recursos materiais}

\section[Planificación]{
	Planificación}

\subsection{Planificación inicial}

\subsection{Seguemento}


	%
% TÍTULO DEL CAPÍTULO
%
\chapter[Análise]{
	Análise
	\label{ch.anl}
}

% 
% SECCION - Título de la sección
%
\section[Requisitos]{
	Requisitos}

\section[Casos de uso]{
	Casos de uso}

	%
% TÍTULO DEL CAPÍTULO
%
\chapter[Deseño]{
	Deseño
	\label{ch.dsn}
}


%
% SECCION - Título de la sección
%
\section[Arquitectura Global do Sistema]{
	Arquitectura do sistema}

\section[Deseño]{
	Deseño}
	%
% TÍTULO DEL CAPÍTULO
%
\chapter[Implementación e probas]{
	Implementación e probas
	\label{ch.iep}
}





	%
% TÍTULO DEL CAPÍTULO
%
\chapter[Solución desenvolta]{
	Solución desenvolta
	\label{ch.sdv.web}
}

%
% SECCION - Título de la sección
%
\section[Introdución]{
	Introdución}

	%
% TÍTULO DEL CAPÍTULO
%
\chapter[Conclusións e traballo futuro]{
	Conclusións e traballo futuro
	\label{ch.cet}
}

%
% SECCION - Título de la sección
%
\section[Conclusións]{
	Conclusións}

\section[Traballo futuro]{
	Traballo futuro}


	% INCLUIMOS LOS AP�NDICES...
    	\appendix
		%
% Memoria económica.
%
\chapter[Manual de usuario]{Manual de usuario}

%		\include{pfc_appendix_020}


	% INCLUIMOS LA BIBLIOGRAF�A...
	\nocite{*}	% Se usa para indicar en la bibliograf�a las referencias no citadas.
	\bibliography{pfc_biblio.bib}
	\bibliographystyle{ieeetr}

\end{document}

